\documentclass[../../main.tex]{subfiles}

\begin{document}

  \begin{shaded}
    Let $f: \R \to \R$ be a function such that $(f(x))^n$ is a polynomial for every $n \geq 2$. Does it follow that $f$ is a polynomial?
  \end{shaded}

  \textbf{Solution:}

  The answer is: yes.

  Let $\displaystyle\frac{f^3}{f^2} = \displaystyle\frac{p}{q}$ be an irreducible rational function with som polynomials $p$ and $q$. Then $\displaystyle\frac{p^2}{q^2}$ is also irreducible, and it is satisfied that $\displaystyle\frac{p^2}{q^2} = \left(\displaystyle\frac{f^3}{f^2}\right)^2 = f^2$. Since $f^2$ is a polynomial, then $q$ is constant (due to the fact that $q^2$ is constant too).

  Hence, $f = \displaystyle\frac{f^3}{f^2} = \displaystyle\frac{p}{q}$ is a polynomial $p$ divided by a constant $q$, and then it is clear that $f$ is a polynomial.

\end{document}
