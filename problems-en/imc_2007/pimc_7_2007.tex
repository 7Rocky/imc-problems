\documentclass[../../main.tex]{subfiles}

\begin{document}

  \begin{shaded}
    Let $f: \R \to \R$ be a continuous function. Suppose that for any $c > 0$, the graph of $f$ can be moved to the graph of $c f$ using only a translation or a rotation. Does this imply that $f(x) = a x + b$, for some real numbers $a$ and $b$?
  \end{shaded}

  \textbf{Solution:}

  No. If $f(x) = \e^x$, then $c f(x) = c \, \e^x = \e^{x + \ln{c}} = f(x + \ln{c})$, which is a translation of magnitude $\ln{c}$ over the X axis.

\end{document}
