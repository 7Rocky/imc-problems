\documentclass[../../main.tex]{subfiles}

\begin{document}

  \begin{shaded}
    A four-digit number $YEAR$ is called \textit{very good} if the system
    $$
    \left \{
      \begin{matrix}
        Y x + E y + A z + R w & = & Y \\
        R x + Y y + E z + A w & = & E \\
        A x + R y + Y z + E w & = & A \\
        E x + A y + R z + Y w & = & R
      \end{matrix}
    \right .
    $$

    of linear equations in the variables $x$, $y$, $z$ and $w$ has at least two solutions. Find all very good $YEAR$s in the 21st century (the 21st century starts in 2001 and ends in 2100).
  \end{shaded}

  \textbf{Solution:}

  Let $M$, $X$ and $N$ be the following matrices:
  $$
  M = \left(
    \begin{matrix}
      Y & E & A & R \\
      R & Y & E & A \\
      A & R & Y & E \\
      E & A & R & Y
    \end{matrix}
  \right) \quad\quad X =
  \left(
    \begin{matrix}
      x \\
      y \\
      z \\
      w
    \end{matrix}
  \right) \quad\quad N =
  \left(
    \begin{matrix}
      Y \\
      E \\
      A \\
      R
    \end{matrix}
  \right)
  $$

  The system of linear equations $M X = N$ has a unique solution if and only if $\det{(M)} \ne 0$. The system can have more than one solution (indeed, infinite) if $\det{(M)} = 0$ and also $\rank{(M)} = \rank{(M | N)}$. It could also happen that the system does not have any solution when $\rank{(M)} < \rank{(M | N)}$. All of these comes from the Rouché-Capelli theorem.

  The cases we are interested in are the ones that make $\det{(M)} = 0$ y $\rank{(M)} = \rank{(M | N)}$. We can notice that years in the 21st century satisfy $Y = 2$ always and $E = 0$ except for $2100$. This case can be checked particularly:
  $$
  \det{(M)} = \left|
    \begin{matrix}
      2 & 1 & 0 & 0 \\
      0 & 2 & 1 & 0 \\
      0 & 0 & 2 & 1 \\
      1 & 0 & 0 & 2
    \end{matrix}
  \right| = 2 \cdot \left |
    \begin{matrix}
      2 & 1 & 0 \\
      0 & 2 & 1 \\
      0 & 0 & 2
    \end{matrix}
  \right| - \left|
    \begin{matrix}
      1 & 0 & 0 \\
      2 & 1 & 0 \\
      0 & 2 & 1 \\
    \end{matrix}
  \right| = 15 \ne 0
  $$

  Now we can discard year $2100$, hence, we can fix $Y = 2$ and $E = 0$. Now we proceed to calculate the determinant of the coefficients matrix depending on $A$ and $R$:

  \begin{equation*}
    \begin{split}
      \det{(M)} & = \left|
        \begin{matrix}
          2 & 0 & A & R \\
          R & 2 & 0 & A \\
          A & R & 2 & 0 \\
          0 & A & R & 2
        \end{matrix}
      \right| = (2 + A + R) \left|
        \begin{matrix}
          1 & 1 & 1 & 1 \\
          R & 2 & 0 & A \\
          A & R & 2 & 0 \\
          0 & A & R & 2
        \end{matrix}
      \right| \\ & = (2 + A + R) \left|
        \begin{matrix}
          1 &   0   &   0   &   0   \\
          R & 2 - R &  - R  & A - R \\
          A & R - A & 2 - A &  - A  \\
          0 &   A   &   R   &   2
        \end{matrix}
      \right| \\ & = (2 + A + R) \left|
        \begin{matrix}
          2 - R &  - R  & A - R \\
          R - A & 2 - A &  - A  \\
            A   &   R   &   2
        \end{matrix}
      \right| \\ & = (2 + A + R) \left|
        \begin{matrix}
          2 + A - R &   0   & 2 + A - R \\
            R - A   & 2 - A &    - A    \\
              A     &   R   &     2
        \end{matrix}
      \right| \\ & = (2 + A + R)(2 + A - R) \left|
        \begin{matrix}
            1   &   0   &  1  \\
          R - A & 2 - A & - A \\
            A   &   R   &  2
        \end{matrix}
      \right| \\ & = (2 + A + R)(2 + A - R) \left|
        \begin{matrix}
            1   &   0   &   0  \\
          R - A & 2 - A &  - R \\
            A   &   R   & 2 - A
        \end{matrix}
      \right| \\ & = (2 + A + R)(2 + A - R) \left|
        \begin{matrix}
          2 - A &  - R \\
            R   & 2 - A
        \end{matrix}
      \right| \\ & = (2 + A + R)(2 + A - R)\left((2 - A)^2 + R^2\right)
    \end{split}
  \end{equation*}

  We can observe that $\det{(M)} = 0$ if some of the three factors of the expression is zero. The first factor $(2 + A + R) \geq 2$ since $0 \leq A, R \leq 9$. The second factor is zero when $A = R - 2$, so $R \geq 2$. These cases are $2002, 2013, \dots, 2079$. And the third factor equals zero when $A = 2$ and $R = 0$, that is, year $2020$.

  Having computed the values that make the determinant equal zero, we need to verify that $\rank{(M)} = \rank{(M | N)}$. The third case can be done particularly because it is unique:
  $$
  (M | N) \sim \left(
    \begin{matrix}
      2 & 0 & 2 & 0 & \\
      0 & 2 & 0 & 2 & \\
      2 & 0 & 2 & 0 & \\
      0 & 2 & 0 & 2 &
    \end{matrix}
    \left|
      \begin{matrix}
        & 2 \\
        & 0 \\
        & 2 \\
        & 0
      \end{matrix}
    \right .
  \right)
  \sim
  \left(
    \begin{matrix}
      1 & 0 & 1 & 0 & \\
      0 & 1 & 0 & 1 & \\
      0 & 0 & 0 & 0 & \\
      0 & 0 & 0 & 0 &
    \end{matrix}
    \left|
      \begin{matrix}
        & 1 \\
        & 0 \\
        & 0 \\
        & 0
      \end{matrix}
    \right .
  \right)
  $$

  Since $\rank{(M)} = \rank{(M | N)} = 2$, year $2020$ is a \textit{very good} year.

  For the second case, we need to calculate the rank of the matrices depending on $R$:
  \begin{equation*}
    \begin{split}
      (M | N) & \sim \left(
        \begin{matrix}
            2   &   0   & R - 2 &   R   & \\
            R   &   2   &   0   & R - 2 & \\
          R - 2 &   R   &   2   &   0   & \\
            0   & R - 2 &   R   &   2   &
        \end{matrix}
        \left|
          \begin{matrix}
            &   2   \\
            &   0   \\
            & R - 2 \\
            &   R
          \end{matrix}
        \right .
      \right) \\ & \sim \left(
        \begin{matrix}
            R   &   R   & R &   R   & \\
            R   &   2   & 0 & R - 2 & \\
          R - 2 &   R   & 2 &   0   & \\
            0   & R - 2 & R &   2   &
        \end{matrix}
        \left|
          \begin{matrix}
            &   R   \\
            &   0   \\
            & R - 2 \\
            &   R
          \end{matrix}
        \right .
      \right) \\ & \sim \left(
        \begin{matrix}
          1 &   1   &   1   &   1   & \\
          0 & 2 - R &  - R  &  - 2  & \\
          0 &   2   & 4 - R & 2 - R & \\
          0 & R - 2 &   R   &   2   &
        \end{matrix}
        \left|
          \begin{matrix}
            &  1  \\
            & - R \\
            &  0  \\
            &  R
          \end{matrix}
        \right .
      \right) \\ & \sim \left(
        \begin{matrix}
          1 &   1   &  1  &   1   & \\
          0 & 2 - R & - R &  - 2  & \\
          0 &   R   &  4  & 4 - R & \\
          0 &   0   &  0  &   0   &
        \end{matrix}
        \left|
          \begin{matrix}
            &  1  \\
            & - R \\
            &  R  \\
            &  0
          \end{matrix}
        \right .
      \right) \\ & \sim \left(
        \begin{matrix}
          1 &   1   &  1  &   1   & \\
          0 & 1  & 2 - \frac{R}{2} &  1 - \frac{R}{2}  & \\
          0 &   R   &  4  & 4 - R & \\
          0 &   0   &  0  &   0   &
        \end{matrix}
        \left|
          \begin{matrix}
            & 1 \\
            & 0 \\
            & R \\
            & 0
          \end{matrix}
        \right .
      \right) \\ & \sim \left(
        \begin{matrix}
          1 &   1   &  1  &   1   & \\
          0 & 1  & 2 - \frac{R}{2} &  1 - \frac{R}{2}  & \\
          0 &   0   &  4 - 2R + \frac{R^2}{2}  & 4 - 2R + \frac{R^2}{2} & \\
          0 &   0   &  0  &   0   &
        \end{matrix}
        \left|
          \begin{matrix}
            & 1 \\
            & 0 \\
            & R \\
            & 0
          \end{matrix}
        \right .
      \right)
    \end{split}
  \end{equation*}

  Since $4 - 2R + \displaystyle\frac{R^2}{2} \ne 0$ for integer values of $R$ between $2$ and $9$, we have that $\rank{(M)} = \rank{(M|N)} = 3$.

  To sum up, the list of \textit{very good} years within the 21st century are: $2002$, $2013$, $2020$, $2024$, $2035$, $2046$, $2057$, $2068$ y $2079$.

\end{document}
