\documentclass[../../main.tex]{subfiles}

\pagenumbering{gobble}

\begin{document}

  \begin{shaded}
    Sean $(a_n)_{n = 1}^\infty$ y $(b_n)_{n = 1}^\infty$ dos sucesiones de números positivos. Demuestra que los siguientes enunciados son equivalentes.

    \begin{enumerate}[(1)]
      \item Existe una sucesión $(c_n)_{n = 1}^\infty$ de números positivos tal que $\displaystyle\sum_{n = 1}^\infty \frac{a_n}{c_n}$ y $\displaystyle\sum_{n = 1}^\infty \frac{c_n}{b_n}$ son convergentes.
      \item $\displaystyle\sum_{n = 1}^\infty \sqrt{\frac{a_n}{b_n}}$ converge.
    \end{enumerate}
  \end{shaded}

  \textbf{Solución:}

  Se trata de un ``si y solo si'', es decir, hay que demostrar dos implicaciones:

  (1) $\Longleftarrow$ (2):

  La expresión radical del enunciado (2) se puede expresar de las siguientes maneras:
  $$
  \sqrt{\frac{a_n}{b_n}} = \sqrt{\frac{a_n}{b_n}} \cdot \sqrt{\frac{a_n}{a_n}} = \frac{a_n}{\sqrt{a_n \cdot b_n}} = \frac{a_n}{c_n}
  $$
  $$
  \sqrt{\frac{a_n}{b_n}} = \sqrt{\frac{a_n}{b_n}} \cdot \sqrt{\frac{b_n}{b_n}} = \frac{\sqrt{a_n \cdot b_n}}{b_n} = \frac{c_n}{b_n}
  $$

  Si se elige $c_n = \sqrt{a_n \cdot b_n}$, y sabiendo que $\displaystyle\sum_{n = 1}^\infty \sqrt{\frac{a_n}{b_n}}$ converge, entonces seguro que $\displaystyle\sum_{n = 1}^\infty \frac{a_n}{c_n}$ y $\displaystyle\sum_{n = 1}^\infty \frac{c_n}{b_n}$ son convergentes; y por tanto existe una sucesión $(c_n)_{n = 1}^\infty$ que hace que ambas sumas converjan.

  (1) $\Longrightarrow$ (2):

  Usando la desigualdad de Cauchy-Schwarz:
  $$
  \left(\sum _{k = 1}^n \alpha_k \cdot \beta_k\right)^2 \leq \left(\sum _{k = 1}^n \alpha_k^2\right)\left(\sum _{k=1}^n \beta_k^2\right)
  $$

  Y asignando $\alpha_k = \displaystyle\sqrt{\frac{a_k}{c_k}}$ y $\beta_k = \displaystyle\sqrt{\frac{c_k}{b_k}}$, se obtiene lo siguiente
  \begin{equation*}
    \begin{split}
      & 0 < \left(\sum_{n = 1}^\infty \sqrt{\frac{a_n}{c_n}} \cdot \sqrt{\frac{c_n}{b_n}}\right)^2 \leq \left(\sum_{n = 1}^\infty \left(\sqrt{\frac{a_n}{c_n}}\right)^2\right)
      \cdot
      \left(\sum_{n = 1}^\infty \left(\sqrt{\frac{c_n}{b_n}}\right)^2\right) \\ \iff &
      0 < \left(\sum_{n = 1}^\infty \sqrt{\frac{a_n}{b_n}}\right)^2 \leq \left(\sum_{n = 1}^\infty \frac{a_n}{c_n}\right)
      \cdot
      \left(\sum_{n = 1}^\infty \frac{c_n}{b_n}\right) \\ \iff &
      0 < \sum_{n = 1}^\infty \sqrt{\frac{a_n}{b_n}} \leq \sqrt{\left(\sum_{n = 1}^\infty \frac{a_n}{c_n}\right)
      \cdot
      \left(\sum_{n = 1}^\infty \frac{c_n}{b_n}\right)}
    \end{split}
  \end{equation*}

  Como la serie $\displaystyle\sum_{n = 1}^\infty \sqrt{\frac{a_n}{b_n}}$ está acotada superiormente por un valor finito (e inferiormente, al tratarse de una sucesión de números positivos), se demuestra que es convergente. \\

  Y con esto, queda demostrado que (1) $\iff$ (2).

\end{document}
