\documentclass[../../main.tex]{subfiles}

\begin{document}

  \begin{shaded}
    Sea $f: [0, \infty) \longrightarrow \R$ una función continua tal que existe $\displaystyle\lim_{x \to \infty} f(x) = L$ (puede ser finito o infinito). Prueba que
    $$
    \lim_{n \to \infty} \int_0^1 f(nx) \dx = L
    $$
  \end{shaded}

  \textbf{Solución:}

  Realizando un cambio de variable, se obtiene lo siguiente:
  $$
  \left .
    \begin{matrix}
      t = nx \\
      \dt = n \dx
    \end{matrix}
  \right \}
  \Longrightarrow
  \int_0^1 f(nx) \dx = \frac{1}{n} \cdot \int_0^n f(t) \dt
  $$

  Utilizando ahora la Regla de L'Hôpital, se demuestra el resultado:
  $$
  \lim_{n \to \infty} \int_0^1 f(nx) \dx = \lim_{n \to \infty} \frac{1}{n} \cdot \int_0^n f(t) \dt = \lim_{n \to \infty} \frac{\displaystyle\frac{\mathrm{d}}{\mathrm{d}n}\bigg(\displaystyle\int_0^n f(t) \dt\bigg)}{\displaystyle\frac{\mathrm{d}}{\mathrm{d}n}(n)} = \lim_{n \to \infty} f(n) = L
  $$

\end{document}
