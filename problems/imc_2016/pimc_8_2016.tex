\documentclass[../../main.tex]{subfiles}

\begin{document}

  \begin{shaded}
    Sea $n$ un número entero positivo, y sea $\Z_n$ el anillo de enteros módulo $n$. Supóngase que existe una función $f: \Z_n \to \Z_n$ que satisface las siguientes propiedades:

    \begin{enumerate}
      \item $f(x) \ne x$,
      \item $f(f(x)) = x$,
      \item $f(f(f(x + 1) + 1) + 1) = x$, para todo $x \in \Z_n$
    \end{enumerate}

    Prueba que $n \equiv 2 \pmod{4}$.
  \end{shaded}

  \textbf{Solución:}

  En primer lugar, se ve que $f$ es una permutación sobre los elementos de $\Z_n$, sin ningún punto fijo, por la primera propiedad.

  Una permutación se puede descomponer en un producto de transposiciones. En este caso, las transposiciones de la permutación $f$ son de la forma $\left(x, f(x)\right)$, de manera que $f \circ f$ es la identidad. Por este motivo, las transposiciones son disjuntas, lo cual indica que el número de transposiciones es $\displaystyle\frac{n}{2}$ y por tanto $n$ es un número par.

  Sea $g(x) = f(x + 1)$. Si $g$ fuera una permutación impar, entonces $g \circ g \circ g$ es impar, pero $g(g(g(x))) = g(g(f(x + 1))) = g(f(f(x + 1) + 1)) = f(f(f(x + 1) + 1) + 1) = x$, por la tercera propiedad de $f$. Como la permutación identidad es par, se contradice que $g$ es impar, por lo que $g$ es par.

  Considerando la permutación cíclica $h(x) = x - 1$, esta es impar porque el número de elementos no fijos es $n$, que es un número par.

  Entonces, $f = g \circ h$ es una permutación impar, lo cual implica que el número de transposiciones en las que se puede descomponer la permutación es impar, es decir, que $\displaystyle\frac{n}{2}$ es impar. Finalmente, esto es lo mismo que decir que $n \equiv 2 \pmod{4}$.

\end{document}
