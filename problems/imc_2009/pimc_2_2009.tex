\documentclass[../../main.tex]{subfiles}

\begin{document}

  \begin{shaded}
    Sean $A$, $B$ y $C$ matrices cuadradas reales del mismo orden. Se sabe que $A$ es invertible. Demuestra que si $(A - B) C = B A^{-1}$ entonces $C (A - B) = A^{-1} B$.
  \end{shaded}

  \textbf{Solución:}

  Modificando la expresión que se cumple, se obtiene lo siguiente:
  \begin{equation*}
    \begin{split}
      (A - B) C = B A^{-1} & \iff 
      (A - B) C A = B \\ & \iff
      A^{-1} (A - B) C A = A^{-1} B \\ & \iff
      (I - A^{-1} B) C A = A^{-1} B \\ & \iff
      C A - A^{-1} B C A = A^{-1} B
    \end{split}
  \end{equation*}

  Para que se cumpla que $C (A - B) = A^{-1} B$, basta con demostrar que $A^{-1} B C A = C B$, lo cual se puede ver suponiendo que se cumple que $C (A - B) = A^{-1} B$. Entonces:
  $$
  (A - B) C = B A^{-1} \iff (A - B) C A = B \iff A C A - B C A = B
  $$
  $$
  C (A - B) = A^{-1} B \iff A C (A - B) = B \iff A C A - A C B = B
  $$

  Uniendo estas dos expresiones se obtiene $B C A = A C B \iff A^{-1} B C A = C B$, con lo cual queda demostrado que si $(A - B) C = B A^{-1}$ entonces $C (A - B) = A^{-1} B$.

  Otra manera de verlo:
  \begin{equation*}
    \begin{split}
      (A - B) C = B A^{-1} & \iff 
      A C - B C - B A^{-1} + I = I \\ & \iff
      A C - B C + A A^{-1} - B A^{-1} = I \\ & \iff
      (A - B) (C + A^{-1}) = I \\ & \iff
      (C + A^{-1}) (A - B) = I \\ & \iff
      C A + A^{-1} A - C B - A^{-1} B = I \\ & \iff
      C A + I - C B - A^{-1} B = I \\ & \iff
      C (A - B) = A^{-1} B
    \end{split}
  \end{equation*}

\end{document}
