\documentclass[../../main.tex]{subfiles}

\begin{document}

  \begin{shaded}
    Se define la sucesión de números $a_0, a_1, \dots$ con la siguiente recurrencia:
    $$
    a_0 = 1, \quad a_1 = 2, \quad (n + 3) a_{n+2} = (6n + 9) a_{n+1} - n a_n \quad \forall n \geq 0
    $$

    Prueba que todos los términos de la sucesión son números enteros.
  \end{shaded}

  \textbf{Solución:}

  En este ejercicio parece bastante conveniente el uso de la función generatriz de la sucesión, de manera que $f(x) = \displaystyle\sum_{n = 0}^\infty a_n \, x^n$. La derivada de esta función generatriz es $f'(x) = \displaystyle\sum_{n = 0}^\infty n \, a_n \, x^{n - 1}$. Entonces
  \begin{equation*}
    \begin{split}
      f'(x) & =
      \sum_{n = 0}^\infty n \, a_n \, x^{n - 1} = \sum_{n = 1}^\infty n \, a_n \, x^{n - 1} = \\ & =
      \sum_{n = 0}^\infty \left((6n + 9) a_{n + 1} - (n + 3) a_{n + 2}\right) \, x^{n - 1} = \\ & =
      \sum_{n = 0}^\infty 6n a_{n + 1} x^{n - 1} + \sum_{n = 0}^\infty 9 a_{n + 1} x^{n - 1} - \sum_{n = 0}^\infty n a_{n + 2} x^{n - 1} - \sum_{n = 0}^\infty 3 a_{n + 2} x^{n - 1} = \\ & =
      \sum_{n = 1}^\infty 6n a_{n + 1} x^{n - 1} + \sum_{n = 0}^\infty 9 a_{n + 1} x^{n - 1} - \sum_{n = 1}^\infty n a_{n + 2} x^{n - 1} - \sum_{n = 0}^\infty 3 a_{n + 2} x^{n - 1} = \\ & =
      \frac{6}{x^2}\sum_{n = 1}^\infty n a_{n + 1} x^{n + 1} + \frac{9}{x^2} \sum_{n = 0}^\infty a_{n + 1} x^{n + 1} - \frac{1}{x^3} \sum_{n = 1}^\infty n a_{n + 2} x^{n + 2} - \frac{3}{x^3} \sum_{n = 0}^\infty a_{n + 2} x^{n + 2} = \\ & =
      \frac{6}{x^2}\sum_{n = 2}^\infty (n - 1) a_n x^n + \frac{9}{x^2} \sum_{n = 1}^\infty a_n x^n - \frac{1}{x^3} \sum_{n = 3}^\infty (n - 2) a_n x^n - \frac{3}{x^3} \sum_{n = 2}^\infty a_n x^n = \\ & =
      \frac{6}{x^2}\sum_{n = 2}^\infty n a_n x^n - \frac{6}{x^2}\sum_{n = 2}^\infty a_n x^n + \frac{9}{x^2} \sum_{n = 1}^\infty a_n x^n - \frac{1}{x^3} \sum_{n = 3}^\infty n a_n x^n + \\ & \qquad + \frac{2}{x^3} \sum_{n = 3}^\infty  a_n x^n - \frac{3}{x^3} \sum_{n = 2}^\infty a_n x^n =
    \end{split}
  \end{equation*}
  \begin{equation*}
    \begin{split}
      &
      \frac{6}{x}\sum_{n = 2}^\infty n a_n x^{n - 1} - \frac{6}{x^2}\sum_{n = 2}^\infty a_n x^n + \frac{9}{x^2} \sum_{n = 1}^\infty a_n x^n - \frac{1}{x^2} \sum_{n = 3}^\infty n a_n x^{n - 1} + \\ & \qquad + \frac{2}{x^3} \sum_{n = 3}^\infty  a_n x^n - \frac{3}{x^3} \sum_{n = 2}^\infty a_n x^n = \\ & =
      \frac{6}{x} (f'(x) - a_1) - \frac{6}{x^2} (f(x) - a_0 - a_1 x) + \frac{9}{x^2} (f(x) - a_0)) - \frac{1}{x^2} (f'(x) - a_1 - 2 a_2 x) + \\ & \qquad + \frac{2}{x^3} (f(x) - a_0 - a_1 x - a_2 x^2) - \frac{3}{x^3} (f(x) - a_0 - a_1 x)
    \end{split}
  \end{equation*}

  Sabiendo que $a_0 = 1$, $a_1 = 2$ y que $a_2 = 6$, la expresión queda como
  \begin{equation*}
    \begin{split}
      f'(x) & =
      \frac{6}{x} (f'(x) - 2) - \frac{6}{x^2} (f(x) - 1 - 2 x) + \frac{9}{x^2} (f(x) - 1)) - \frac{1}{x^2} (f'(x) - 2 - 12 x) + \\ & \qquad + \frac{2}{x^3} (f(x) - 1 - 2 x - 6 x^2) - \frac{3}{x^3} (f(x) - 1 - 2 x) = \\ & =
      \left(\frac{6}{x} - \frac{1}{x^2}\right) f'(x) + \left(\frac{3}{x^2} - \frac{1}{x^3}\right) f(x) + \left(\frac{1}{x^2} + \frac{1}{x^3}\right)
    \end{split}
  \end{equation*}

  Y entonces se llega a la siguiente ecuación diferencial:
  $$
  \left(1 - \frac{6}{x} + \frac{1}{x^2}\right) f'(x) = \left(\frac{3}{x^2} - \frac{1}{x^3}\right) f(x) + \left(\frac{1}{x^2} + \frac{1}{x^3}\right)
  $$

  Que es equivalente a una EDO lineal:
  $$
  f'(x) + \frac{1 - 3x}{x^3 - 6 x^2 + x} f(x) = \frac{x + 1}{x^3 - 6 x^2 + x}
  $$

  Con $f(0) = a_0 = 1$. La solución de esta EDO es
  $$
  f(x) = \e^{- \int \frac{1 - 3x}{x^3 - 6 x^2 + x} \dx} \cdot \left(C + \int \frac{x + 1}{x^3 - 6 x^2 + x} \cdot \e^{\int \frac{1 - 3x}{x^3 - 6 x^2 +x} \dx} \text{d}x\right)
  $$

  Por un lado,
  \begin{equation*}
    \begin{split}
      \int \frac{1 - 3x}{x^3 - 6 x^2 + x} \dx & =
      \int \left(\frac{1}{x} - \frac{x - 3}{x^2 - 6x + 1}\right) \text{d}x = \\ & =
      \ln{|x|} - \frac{1}{2} \ln{|x^2 - 6x + 1|} = \\ & =
      \ln{\left|\frac{x}{\sqrt{x^2 - 6x + 1}}\right|}
    \end{split}
  \end{equation*}

  Y entonces
  $$
  \e^{- \int \frac{1 - 3x}{x^3 - 6 x^2 + x} \dx} = \frac{\sqrt{x^2 - 6x + 1}}{x}
  $$

  Y, por otro lado
  \begin{equation*}
    \begin{split}
      \int \frac{x + 1}{x^3 - 6 x^2 + x} \cdot \e^{\int \frac{1 - 3x}{x^3 - 6 x^2 +x} \dx} \text{d}x & =
      \int \frac{x + 1}{x^3 - 6 x^2 + x} \cdot \frac{x}{\sqrt{x^2 - 6x + 1}} \dx = \\ & =
      \int \frac{x + 1}{(x^2 - 6 x + 1)^{\frac{3}{2}}} \dx = \\ & =
      \int \frac{x - 3}{(x^2 - 6 x + 1)^{\frac{3}{2}}} \dx + \int \frac{4}{(x^2 - 6 x + 1)^{\frac{3}{2}}} \dx = \\ & =
      \frac{-1}{\sqrt{x^2 - 6x + 1}} + \int \frac{4}{((x - 3)^2 - 8)^{\frac{3}{2}}} \dx
    \end{split}
  \end{equation*}
  \begin{equation*}
    \begin{split}
      \int \frac{4}{((x - 3)^2 - 8)^{\frac{3}{2}}} \dx & =
      \left\{
        \begin{matrix}
          x & = & t + 3 \\
          \text{d}x & = & \text{d}t
        \end{matrix}
      \right\} =
      \int \frac{4}{(t^2 - 8)^{\frac{3}{2}}} \dt = \\ & =
      \left\{
        \begin{matrix}
          t & = & \sqrt{8}\sec{u} \\
          \text{d}t & = & \sqrt{8}\tan{u}\sec{u} \,\text{d}u
        \end{matrix}
      \right\} =
      \int \frac{4 \sqrt{8}\tan{u}\sec{u}}{(8\sec^2{u} - 8)^{\frac{3}{2}}} \,\text{d}u = \\ & =
      \int \frac{4\sqrt{8}\tan{u}\sec{u}}{(\sqrt{8})^3 \tan^3{u}} \,\text{d}u = \frac{1}{2} \int \frac{\sec{u}}{\tan^2{u}} \,\text{d}u = \frac{1}{2} \int \frac{\cos{u}}{\sin^2{u}} \,\text{d}u = \\ & =
      \left\{
        \begin{matrix}
          v & = & \sin{u} \\
          \text{d}v & = & \cos{u} \,\text{d}u
        \end{matrix}
      \right\} =
      \frac{1}{2} \int \frac{\text{d}v}{v^2} = \\ & =
      \frac{-1}{2v} = \frac{-1}{2 \sin{u}} = \frac{-1}{2 \sin{\left(\arccos{\left(\frac{\sqrt{8}}{t}\right)}\right)}} = \\ & =
      \frac{-1}{2 \sqrt{1 - \frac{8}{t^2}}} = \frac{-t}{2 \sqrt{t^2 - 8}} = \\ & =
      \frac{-(x - 3)}{2 \sqrt{(x - 3)^2 - 8}} = \frac{3 - x}{2 \sqrt{x^2 - 6x + 1}}
    \end{split}
  \end{equation*}

  Y por tanto,
  $$
  \int \frac{x + 1}{x^3 - 6 x^2 + x} \cdot \e^{\int \frac{1 - 3x}{x^3 - 6 x^2 +x} \dx} \text{d}x = \frac{-1}{\sqrt{x^2 - 6x + 1}} + \frac{3 - x}{2 \sqrt{x^2 - 6x + 1}} = \frac{1 - x}{2 \sqrt{x^2 - 6x + 1}}
  $$

  Entonces, la solución a la EDO es:
  \begin{equation*}
    \begin{split}
      f(x) & = \e^{- \int \frac{1 - 3x}{x^3 - 6 x^2 + x} \dx} \cdot \left(C + \int \frac{x + 1}{x^3 - 6 x^2 + x} \cdot \e^{\int \frac{1 - 3x}{x^3 - 6 x^2 +x} \dx} \text{d}x\right) = \\ & =
      \frac{\sqrt{x^2 - 6x + 1}}{x} \cdot \left(C + \frac{1 - x}{2 \sqrt{x^2 - 6x + 1}}\right) = \\ & =
      C \cdot\frac{\sqrt{x^2 - 6x + 1}}{x} + \frac{1 - x}{2 x} = \\ & =
      \frac{1 - x + 2 C \sqrt{x^2 - 6x + 1}}{2 x}
    \end{split}
  \end{equation*}

  Sabiendo que $f(0) = 1$, se tiene que
  $$
  f(x) = \frac{1 - x + 2 C \sqrt{x^2 - 6x + 1}}{2 x} \iff 2x \, f(x) = 1 - x + 2 C \sqrt{x^2 - 6x + 1}
  $$
  $$
  2 \cdot 0 \, f(0) = 1 - 0 + 2 C \sqrt{0^2 - 6 \cdot 0 + 1} \iff 0 = 1 + 2C \iff C = \frac{-1}{2}
  $$

  Entonces
  $$
  f(x) = \frac{1 - x - \sqrt{(1 - x)^2 - 4x}}{2 x}
  $$

  De donde se puede ver que $f$ es solución de un polinomio $p(z) = x z^2 + (x - 1) z + 1$, ya que $p(f(x)) = 0$. Y, por tanto
  $$
  x (f(x))^2 + (x - 1) f(x) + 1 = 0
  $$

  Sustituyendo $f(x)$ por $\displaystyle\sum_{n = 0}^\infty a_n \, x^n$, se ve que
  $$
  x \left(\sum_{n = 0}^\infty a_n \, x^n\right)^2 + (x - 1) \sum_{n = 0}^\infty a_n \, x^n + 1 = 0 \iff
  $$
  $$
  x \left(\sum_{n = 0}^\infty a_n \, x^n\right)^2 + \sum_{n = 0}^\infty a_n \, x^{n + 1} - a_0 - \sum_{n = 1}^\infty a_n \, x^n + 1 = 0 \iff
  $$
  $$
  x \left(\sum_{n = 0}^\infty a_n \, x^n\right)^2 + \sum_{n = 0}^\infty a_n \, x^{n + 1} - \sum_{n = 1}^\infty a_n \, x^n = 0 \iff
  $$
  $$
  x \left(\sum_{n = 0}^\infty a_n \, x^n\right)^2 + \sum_{n = 0}^\infty a_n \, x^{n + 1} - \sum_{n = 0}^\infty a_{n + 1} \, x^{n + 1} = 0 \iff
  $$
  $$
  x \left(\sum_{n = 0}^\infty a_n \, x^n\right)^2 = \sum_{n = 0}^\infty (a_{n + 1} - a_n)\, x^{n + 1} \iff
  $$
  $$
  x \sum_{n = 0}^\infty \sum_{k = 0}^n a_k a_{n - k} \, x^n = \sum_{n = 0}^\infty (a_{n + 1} - a_n)\, x^{n + 1} \iff
  $$
  $$
  \sum_{n = 0}^\infty \left(\sum_{k = 0}^n a_k a_{n - k}\right) \, x^{n + 1} = \sum_{n = 0}^\infty (a_{n + 1} - a_n)\, x^{n + 1} \iff
  $$
  $$
  \sum_{k = 0}^n a_k a_{n - k} = a_{n + 1} - a_n \iff
  $$
  $$
  a_{n + 1} = a_n + \sum_{k = 0}^n a_k a_{n - k}
  $$

  Finalmente, se puede ver que las operaciones involucradas para obtener explícitamente los términos de la sucesión $\{a_n\}_{n = 0}^\infty$ son sumas y multiplicaciones. Al ser $a_0 = 1$ y $a_1 = 2$ dos números enteros, para $n \geq 2$, inductivamente, todos los términos serán enteros también.

\end{document}
