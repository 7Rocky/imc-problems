\documentclass[../../main.tex]{subfiles}

\begin{document}

  \begin{shaded}
    Encuentra todas las funciones $f: \mathbb{R} \to (0, + \infty)$ continuas y derivables dos veces que cumplen que
    $$
    f''(x) f(x) \geqslant 2 (f'(x))^2
    $$

    para todo $x \in \mathbb{R}$
  \end{shaded}

  \textbf{Solución:}

  Se puede ver que la desigualdad del enunciado forma parte de una derivada. Por ejemplo, al derivar la siguiente función:
  $$
  g = \frac{1}{f} \Longrightarrow g'' = \left(\frac{1}{f}\right)'' = \left(\frac{-f'}{f^2}\right)' = \frac{-f'' f^2 + 2 f (f')^2}{f^4} = \frac{2 (f')^2 -f'' f}{f^3}
  $$

  Sabiendo que $ 2 (f'(x))^2 - f''(x) f(x) \leqslant 0$, se tiene que
  $$
  \frac{2 (f')^2 -f'' f}{f^3} \geqslant 0 \Longrightarrow g'' \geqslant 0
  $$

  Por lo que la función $g$ es cóncava para todo $x \in \mathbb{R}$. Entonces, para cualquier conjunto de valores $a < b$, $u < a$ y $b < v$, se cumple que
  $$
  \frac{g(a) - g(u)}{a - u} \geqslant \frac{g(b) - g(a)}{b - a} \geqslant \frac{g(v) - g(b)}{v - b}
  $$

  Tomando límites en la expresión anterior cuando $u \to - \infty$ y $v \to + \infty$, se sigue que
  $$
  0 \geqslant \frac{g(b) - g(a)}{b - a} \geqslant 0
  $$

  Como $g = \displaystyle\frac{1}{f}$ es una función estrictamente positiva para todo $x \in \mathbb{R}$, la única posibilidad es que $g(a) = g(b)$ para cualquier $a, b \in \mathbb{R}$. Entonces, $g$ es constante y por tanto $f$ también. Y de aquí se deduce que las únicas funciones que cumplen la desigualdad del enunciado son las funciones constantes positivas, del tipo $f(x) = C$, con $C > 0$.

\end{document}
