\documentclass[../../main.tex]{subfiles}

\begin{document}

  \begin{shaded}
    Sean $A$ y $B$ matrices reales $n \times n$ tales que
    $$
    \rank{(AB - BA + I)} = 1
    $$

    donde $I$ es la matriz identidad $n \times n$. Prueba que
    $$
    \tr{(ABAB)} - \tr{(A^2 B^2)} = \frac{1}{2} n (n - 1)
    $$
  \end{shaded}

  \textbf{Solución:}

  Utilizando algunas propiedades de la traza de dos matrices, como que $\tr{(XY)} = \tr{(YX)}$ y que $\tr{(X)} + \tr{(Y)} = \tr{(X + Y)}$, se tiene que la expresión a demostrar queda como:
  \begin{equation*}
    \begin{split}
      &
      \tr{(ABAB)} - \tr{(A^2 B^2)} = \\ & \qquad =
      \frac{1}{2}\left(2 \, \tr{(ABAB)} - 2 \, \tr{(A^2 B^2)}\right) = \\ & \qquad =
      \frac{1}{2}\left(\tr{(ABAB)} + \tr{(ABAB)} - \tr{(AABB)} - \tr{(AABB)}\right) = \\ & \qquad =
      \frac{1}{2}\left(\tr{(ABAB)} + \tr{(A(BAB))} - \tr{(A(ABB))} - \tr{((AAB)B)}\right) = \\ & \qquad =
      \frac{1}{2}\left(\tr{(ABAB)} + \tr{((BAB)A)} - \tr{((ABB)A)} - \tr{(B(AAB))}\right) = \\ & \qquad =
      \frac{1}{2}\left(\tr{(ABAB)} + \tr{(BABA)} - \tr{(ABBA)} - \tr{(BAAB)}\right) = \\ & \qquad =
      \frac{1}{2}\tr{(ABAB + BABA - ABBA - BAAB)} = \\ & \qquad =
      \frac{1}{2}\tr{((AB - BA)^2)\qquad }
    \end{split}
  \end{equation*}

  Sean $M = AB - BA + I$ y $N = M - I = AB - BA$.

  De la condición $\rank{(M)} = \rank{(AB - BA + I)} = 1$ se obtiene que o bien todos los autovalores de $M$ son nulos con multiplicidad $n - 1$, o bien existe un único autovalor no nulo (con multiplicidad $1$ y el resto de autovalores nulos con multiplicidad $n - 1$).

  Como ocurre que
  \begin{equation*}
    \begin{split}
      \tr{(M)} & =
      \tr{(AB - BA + I)} = \\ & =
      \tr{(AB)} - \tr{(BA)} + \tr{(I)} = \\ & =
      \tr{(I)} = n
    \end{split}
  \end{equation*}

  se tiene que $M$ tiene un único autovalor no nulo con valor $n$ y el resto de autovalores son nulos. Por tanto, el espectro de $M$ es $\sigma(M) = \{0, n\}$.

  Como $N = M - I$, el espectro de $N$ es $\sigma(N) = \{-1, n - 1\}$, y el espectro de $N^2$ es $\sigma(N^2) = \{1, (n - 1)^2\}$.

  Nótese que $\tr{(ABAB)} - \tr{(A^2 B^2)} = \frac{1}{2}\tr{((AB - BA)^2)} = \frac{1}{2}\tr{(N^2)}$. Y se sabe que la traza de $N^2$ es la suma de sus autovalores, por tanto:
  $$
  \tr{(N^2)} = 1 \cdot (n - 1) + (n - 1)^2 = n (n - 1)
  $$

  Y por tanto, queda demostrado que
  $$
  \tr{(ABAB)} - \tr{(A^2 B^2)} = \frac{1}{2}\tr{((AB - BA)^2)} = \frac{1}{2}\tr{(N^2)} = \frac{1}{2} n (n - 1)
  $$

\end{document}
