\documentclass[../../main.tex]{subfiles}

\begin{document}

  \begin{shaded}
    Sea $n$ un número natural. Considérese una matriz $n \times n$ cuyas entradas son $1, 2, \dots, n^2$ escritas en orden empezando de arriba a abajo y de izquierda a derecha. Se eligen $n$ entradas de la matriz de manera que exactamente una entrada es elegida de cada columna y de cada fila. ¿Cuáles son los posibles valores de la suma de las entradas seleccionadas?
  \end{shaded}

  \textbf{Solución:}

  En primer lugar, es conveniente tomar algunos ejemplos:
  $$
  n = 1:
    \begin{pmatrix}
      1
    \end{pmatrix}
  \Longrightarrow S_1 = 1
  $$
  $$
  n = 2:
    \begin{pmatrix}
      1 & 2 \\
      3 & 4
    \end{pmatrix}
  \Longrightarrow S_2 = 5
  $$
  $$
  n = 3:
    \begin{pmatrix}
      1 & 2 & 3 \\
      4 & 5 & 6 \\
      7 & 8 & 9
    \end{pmatrix}
  \Longrightarrow S_3 = 15
  $$

  Es sencillo darse cuenta de la expresión genérica que tiene esta matriz:
  $$
  \begin{pmatrix}
    0n + 1        & 0n + 2        & \dots  & 0n + n        \\
    1n + 1        & 1n + 2        & \dots  & 1n + n        \\
    \vdots        & \vdots        & \ddots & \vdots        \\
    (n - 1) n + 1 & (n - 1) n + 2 & \dots  & (n - 1) n + n
  \end{pmatrix}
  $$

  Como se cogen $n$ entradas de cada fila y columna de manera que no haya filas ni columnas con más de una entrada seleccionada o con ninguna escogida, se ve que la expresión genérica de la suma será:
  $$
  S_n = n \sum_{k = 1}^n (k - 1) + \sum_{k = 1}^n k = n \left(\frac{(n - 1) n}{2}\right) + \frac{(n + 1) n}{2} = \frac{n^3 + n}{2}
  $$

\end{document}
