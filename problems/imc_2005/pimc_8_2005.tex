\documentclass[../../main.tex]{subfiles}

\begin{document}

  \begin{shaded}
    Sea $f: \R \to \R$ una función tal que $(f(x))^n$ es un polinomio para todo $n \geq 2$. Entonces, ¿se cumple que $f$ es un polinomio?
  \end{shaded}

  \textbf{Solución:}

  La respuesta es: sí.
  
  Sea $\displaystyle\frac{f^3}{f^2} = \displaystyle\frac{p}{q}$, una función racional irreducible con $p$ y $q$ dos polinomios. Entonces $\displaystyle\frac{p^2}{q^2}$ también es irreducible, y se cumple que $\displaystyle\frac{p^2}{q^2} = \left(\displaystyle\frac{f^3}{f^2}\right)^2 = f^2$. Como $f^2$ es un polinomio, se tiene que $q$ es constante (ya que $q^2$ lo es).

  Por tanto, $f = \displaystyle\frac{f^3}{f^2} = \displaystyle\frac{p}{q}$ es un polinomio $p$ entre una constante $q$, de donde está claro que $f$ es un polinomio.

\end{document}
