\documentclass[../../main.tex]{subfiles}

\begin{document}

  \begin{shaded}
    Sea $A$ una matriz $n \times n$ tal que sus entradas $a_{ij} = i + j$, siendo $i$ y $j$ los índices de fila y columna, respectivamente. ¿Cuál es el rango de $A$?
  \end{shaded}

  \textbf{Solución:}

  Se puede escribir la matriz $A$ como
  $$
  \begin{pmatrix}
    2 & 3 & \dots & n + 1 \\
    3 & 4 & \dots & n + 2 \\
    \vdots & \vdots & \ddots & \vdots \\
    n + 1 & n + 2 & \dots & n + n
  \end{pmatrix}
  $$

  Al restar la primera fila a todas las demás (excepto la primera fila), se obtiene una matriz con el mismo rango:
  $$
  \begin{pmatrix}
    2 & 3 & \dots & n + 1 \\
    1 & 1 & \dots & 1 \\
    2 & 2 & \dots & 2 \\
    \vdots & \vdots & \ddots & \vdots \\
    n - 1 & n - 1 & \dots & n - 1
  \end{pmatrix}
  $$

  Y esta matriz tiene varias filas que son proporcionales a la segunda fila, por lo que es equivalente a la matriz
  $$
  \begin{pmatrix}
    2 & 3 & \dots & n + 1 \\
    1 & 1 & \dots & 1 \\
    0 & 0 & \dots & 0 \\
    \vdots & \vdots & \ddots & \vdots \\
    0 & 0 & \dots & 0
  \end{pmatrix}
  $$

  Esta última matriz tiene rango $2$, y como es equivalente a la matriz $A$, se demuestra que el rango de la matriz $A$ es $2$.

\end{document}
