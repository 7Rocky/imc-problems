\documentclass[../../main.tex]{subfiles}

\begin{document}

  \begin{shaded}
    Para todo número natural $n \geq 2$ y dos matrices $n \times n$ con entradas reales, $A$ y $B$ que satisfacen la siguiente ecuación:
    $$
    A^{-1} + B^{-1} = (A + B)^{-1}
    $$

    Probar que $\det{(A)} = \det{(B)}$.

    ¿Se puede concluir lo mismo para matrices con entradas complejas?
  \end{shaded}

  \textbf{Solución:}

  Si se multiplica la expresión matricial por $(A + B)$, se obtiene:

  \begin{equation*}
    \begin{split}
      A^{-1} + B^{-1} = (A + B)^{-1} & \iff
      (A^{-1} + B^{-1}) \cdot (A + B) = (A + B)^{-1} \cdot (A + B) \\ & \iff
      I + B^{-1} A + A^{-1} B + I = I \\ & \iff
      A = B (-I - A^{-1} B) \\ & \iff
      A = -B (B^{-1} + A^{-1}) B \\ & \iff
      A = -B (A + B)^{-1} B
    \end{split}
  \end{equation*}

  Y, por otro lado,

  \begin{equation*}
    \begin{split}
      A^{-1} + B^{-1} = (A + B)^{-1} & \iff
      (A^{-1} + B^{-1}) \cdot (A + B) = (A + B)^{-1} \cdot (A + B) \\ & \iff
      I + B^{-1} A + A^{-1} B + I = I \\ & \iff
      B = A (-I - B^{-1} A) \\ & \iff
      B = -A (A^{-1} + B^{-1}) A \\ & \iff
      B = -A (A + B)^{-1} A
    \end{split}
  \end{equation*}

  Estas dos expresiones de $A$ y $B$ son muy similares. Si se toman determinantes se puede ver lo siguiente:
  $$
  \det{(A)} = \det{\left( -B (A + B)^{-1} B\right)} = \frac{(-1)^n \cdot \det{(B)}^2}{\det{(A + B)}}
  $$
  $$
  \det{(B)} = \det{\left(-A (A + B)^{-1} A\right)} = \frac{(-1)^n \cdot \det{(A)}^2}{\det{(A + B)}}
  $$

  Y, uniendo estas dos expresiones tan parecidas:
  $$
  \frac{\det{(A)}}{\det{(B)}^2} = \frac{(-1)^n}{\det{(A + B)}} = \frac{\det{(B)}}{\det{(A)}^2} \iff \left(\frac{\det{(A)}}{\det{(B)}}\right)^3 = 1 \iff \frac{\det{(A)}}{\det{(B)}} = \sqrt[3]{1}
  $$

  Como $\sqrt[3]{1} = 1$ en $\R$, se demuestra que $\det{(A)} = \det{(B)}$ para matrices reales. Sin embargo, en $\mathbb{C}$, $\sqrt[3]{1} = \left\{ 1, \e^{\text{i}\frac{2\pi}{3}}, \e^{-\text{i}\frac{2\pi}{3}}\right\}$, por lo que puede ocurrir que $\det{(A)} \ne \det{(B)}$ en matrices complejas. Lo único que se puede asegurar es que $|\det{(A)}| = |\det{(B)}|$.

\end{document}
