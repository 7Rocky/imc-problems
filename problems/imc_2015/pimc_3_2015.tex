\documentclass[../../main.tex]{subfiles}

\begin{document}

  \begin{shaded}
    Sea $F(0) = 0$, $F(1) = \displaystyle\frac{3}{2}$ y $F(n) = \displaystyle\frac{5}{2} \cdot F(n - 1) - F(n - 2)$ para $n \geq 2$. Determina si $\displaystyle\sum_{n = 0}^\infty \frac{1}{F\left(2^n\right)}$ es un número racional o no.
  \end{shaded}

  \textbf{Solución:}

  En primer lugar, puede ser útil encontrar una expresión de $F(n)$ en función de $n$, sin recurrencias. Para ello, hay que hallar las raíces de su polinomio característico:
  $$
  F(n) = \frac{5}{2} \cdot F(n - 1) - F(n - 2) \Longrightarrow r^n = \frac{5}{2} \cdot r^{n - 1} - r^{n - 2} \Longrightarrow r^2 = \frac{5}{2} \cdot r - 1 \Longrightarrow r = \left\{ \begin{matrix} 2 \\ \textstyle\frac{1}{2} \end{matrix} \right .
  $$

  Por tanto, $F(n) = a \cdot 2^n + b \cdot \left(\displaystyle\frac{1}{2}\right)^n$. Y tomando $n = 0$ y $n = 1$, se ve fácilmente que $a = 1$ y $b = -1$, por lo que $F(n) = 2^n - \displaystyle\frac{1}{2^n}$

  Entonces, $F(2^n) = 2^{2^n} - \displaystyle\frac{1}{2^{2^n}} = \displaystyle\frac{\left(2^{2^n}\right)^2 - 1}{2^{2^n}}$. Para simplificar, $v = 2^n$, y la expresión anterior queda como: $F(v) = \displaystyle\frac{2^{2v} - 1}{2^v}$. Y, por tanto:
  $$
  \sum_{n = 0}^\infty \frac{1}{F(v)} = \sum_{n = 0}^\infty \frac{2^v}{2^{2v} - 1}
  $$

  Es posible hallar el valor de la serie:
  \begin{equation*}
    \begin{split}
      \sum_{n = 0}^\infty \frac{1}{F(v)} & =
      \sum_{n = 0}^\infty \frac{2^v}{2^{2v} - 1} = \sum_{n = 0}^\infty \frac{2^v + 1 - 1}{(2^v + 1)(2^v - 1)} = \\ & =
      \sum_{n = 0}^\infty \left(\frac{1}{2^v - 1} - \frac{1}{(2^v + 1)(2^v - 1)}\right) = \sum_{n = 0}^\infty \left(\frac{1}{2^v - 1} - \frac{1}{2^{2v} - 1}\right) = \\ & =
      \sum_{n = 0}^\infty \left(\frac{1}{2^{2^n} - 1} - \frac{1}{2^{2 \cdot 2^n} - 1}\right) = \sum_{n = 0}^\infty \left(\frac{1}{2^{2^n} - 1} - \frac{1}{2^{2^{n + 1}} - 1}\right) = \\ & =
      \frac{1}{2^{2^0} - 1} - \lim_{n \to \infty} \frac{1}{2^{2^{n + 1}} - 1} = \frac{1}{2^1 - 1} - \frac{1}{\infty} = 1 - 0 = 1
    \end{split}
  \end{equation*}

  Evidentemente, $1 \in \mathbb{Q}$. Y con esto, queda todo perfectamente demostrado.

\end{document}
