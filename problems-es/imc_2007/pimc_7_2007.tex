\documentclass[../../main.tex]{subfiles}

\begin{document}

  \begin{shaded}
    Sea $f: \R \to \R$ una función continua tal que para cualquier $c > 0$, la gráfica de $f$ se puede mover a la gráfica de $c f$ con una rotación o traslación. ¿Implica esto que $f(x) = a x + b$, con $a, b \in \R$?
  \end{shaded}

  \textbf{Solución:}

  No tiene por qué. Si $f(x) = \e^x$, se cumple que $c f(x) = c \, \e^x = \e^{x + \ln{c}} = f(x + \ln{c})$, lo cual supone una traslación horizontal de magnitud $\ln{c}$.

\end{document}
