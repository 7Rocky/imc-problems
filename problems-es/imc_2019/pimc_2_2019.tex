\documentclass[../../main.tex]{subfiles}

\begin{document}

  \begin{shaded}
    Un número de cuatro cifras $YEAR$ es denominado \textit{muy bueno} si el siguiente sistema de ecuaciones en las variables $x$, $y$, $z$ y $w$ tiene al menos dos soluciones:
    $$
    \left \{
      \begin{matrix}
        Y x + E y + A z + R w & = & Y \\
        R x + Y y + E z + A w & = & E \\
        A x + R y + Y z + E w & = & A \\
        E x + A y + R z + Y w & = & R
      \end{matrix}
    \right .
    $$

    Encuentra todos los números $YEAR$ \textit{muy buenos} del siglo XXI (es decir, desde $2001$ hasta $2100$, ambos inclusive).
  \end{shaded}

  \textbf{Solución:}

  Sean las siguientes matrices:
  $$
  M = \left(
    \begin{matrix}
      Y & E & A & R \\
      R & Y & E & A \\
      A & R & Y & E \\
      E & A & R & Y
    \end{matrix}
  \right) \quad\quad X =
  \left(
    \begin{matrix}
      x \\
      y \\
      z \\
      w
    \end{matrix}
  \right) \quad\quad N =
  \left(
    \begin{matrix}
      Y \\
      E \\
      A \\
      R
    \end{matrix}
  \right)
  $$

  El sistema de ecuaciones lineales $M X = N$ tiene una única solución si y solo si $\det{(M)} \ne 0$. El sistema puede tener más de una solución (de hecho, infinitas) si $\det{(M)} = 0$ y además $\rank{(M)} = \rank{(M | N)}$. También podría ocurrir que el sistema no tuviera solución, cuando $\rank{(M)} < \rank{(M | N)}$. Esto se debe al teorema de Rouché-Frobenius.

  Los casos que interesan son los que hacen que $\det{(M)} = 0$ y $\rank{(M)} = \rank{(M | N)}$. Se puede apreciar que los años del siglo XXI cumplen que $Y = 2$ en todos los casos, y que $E = 0$, salvo en el año $2100$. Este caso se puede comprobar de manera particular:
  $$
  \det{(M)} = \left|
    \begin{matrix}
      2 & 1 & 0 & 0 \\
      0 & 2 & 1 & 0 \\
      0 & 0 & 2 & 1 \\
      1 & 0 & 0 & 2
    \end{matrix}
  \right| = 2 \cdot \left |
    \begin{matrix}
      2 & 1 & 0 \\
      0 & 2 & 1 \\
      0 & 0 & 2
    \end{matrix}
  \right| - \left|
    \begin{matrix}
      1 & 0 & 0 \\
      2 & 1 & 0 \\
      0 & 2 & 1 \\
    \end{matrix}
  \right| = 15 \ne 0
  $$

  Visto esto, se descarta el año $2100$, entonces se puede asegurar que $Y = 2$ y $E = 0$. Ahora se calcula el determinante de la matriz de coeficientes en función de $A$ y $R$:

  \begin{equation*}
    \begin{split}
      \det{(M)} & = \left|
        \begin{matrix}
          2 & 0 & A & R \\
          R & 2 & 0 & A \\
          A & R & 2 & 0 \\
          0 & A & R & 2
        \end{matrix}
      \right| = (2 + A + R) \left|
        \begin{matrix}
          1 & 1 & 1 & 1 \\
          R & 2 & 0 & A \\
          A & R & 2 & 0 \\
          0 & A & R & 2
        \end{matrix}
      \right| \\ & = (2 + A + R) \left|
        \begin{matrix}
          1 &   0   &   0   &   0   \\
          R & 2 - R &  - R  & A - R \\
          A & R - A & 2 - A &  - A  \\
          0 &   A   &   R   &   2
        \end{matrix}
      \right| \\ & = (2 + A + R) \left|
        \begin{matrix}
          2 - R &  - R  & A - R \\
          R - A & 2 - A &  - A  \\
            A   &   R   &   2
        \end{matrix}
      \right| \\ & = (2 + A + R) \left|
        \begin{matrix}
          2 + A - R &   0   & 2 + A - R \\
            R - A   & 2 - A &    - A    \\
              A     &   R   &     2
        \end{matrix}
      \right| \\ & = (2 + A + R)(2 + A - R) \left|
        \begin{matrix}
            1   &   0   &  1  \\
          R - A & 2 - A & - A \\
            A   &   R   &  2
        \end{matrix}
      \right| \\ & = (2 + A + R)(2 + A - R) \left|
        \begin{matrix}
            1   &   0   &   0  \\
          R - A & 2 - A &  - R \\
            A   &   R   & 2 - A
        \end{matrix}
      \right| \\ & = (2 + A + R)(2 + A - R) \left|
        \begin{matrix}
          2 - A &  - R \\
            R   & 2 - A
        \end{matrix}
      \right| \\ & = (2 + A + R)(2 + A - R)\left((2 - A)^2 + R^2\right)
    \end{split}
  \end{equation*}

  Se puede ver que $\det{(M)} = 0$ si alguno de los tres factores de la expresión es nula. El primer factor $(2 + A + R) \geq 2$, ya que $0 \leq A, R \leq 9$. El segundo factor se anula cuando $A = R - 2$, por lo que $R \geq 2$. Estos casos son los años $2002, 2013, \dots, 2079$. Y el tercer factor se anula cuando $A = 2$ y $R = 0$, es decir, solo el año $2020$.

  Vistos estos valores que anulan el determinante de la matriz de coeficientes, ahora hay que verificar que $\rank{(M)} = \rank{(M | N)}$. El tercer caso, al ser un único caso puede ser comprobado de manera particular:
  $$
  (M | N) \sim \left(
    \begin{matrix}
      2 & 0 & 2 & 0 & \\
      0 & 2 & 0 & 2 & \\
      2 & 0 & 2 & 0 & \\
      0 & 2 & 0 & 2 &
    \end{matrix}
    \left|
      \begin{matrix}
        & 2 \\
        & 0 \\
        & 2 \\
        & 0
      \end{matrix}
    \right .
  \right)
  \sim
  \left(
    \begin{matrix}
      1 & 0 & 1 & 0 & \\
      0 & 1 & 0 & 1 & \\
      0 & 0 & 0 & 0 & \\
      0 & 0 & 0 & 0 &
    \end{matrix}
    \left|
      \begin{matrix}
        & 1 \\
        & 0 \\
        & 0 \\
        & 0
      \end{matrix}
    \right .
  \right)
  $$

  Como $\rank{(M)} = \rank{(M | N)} = 2$, el año $2020$ es un año \textit{muy bueno}.

  Y para el segundo caso, se han de calcular los rangos de las matrices de coeficientes y ampliada en función de $R$:
  \begin{equation*}
    \begin{split}
      (M | N) & \sim \left(
        \begin{matrix}
            2   &   0   & R - 2 &   R   & \\
            R   &   2   &   0   & R - 2 & \\
          R - 2 &   R   &   2   &   0   & \\
            0   & R - 2 &   R   &   2   &
        \end{matrix}
        \left|
          \begin{matrix}
            &   2   \\
            &   0   \\
            & R - 2 \\
            &   R
          \end{matrix}
        \right .
      \right) \\ & \sim \left(
        \begin{matrix}
            R   &   R   & R &   R   & \\
            R   &   2   & 0 & R - 2 & \\
          R - 2 &   R   & 2 &   0   & \\
            0   & R - 2 & R &   2   &
        \end{matrix}
        \left|
          \begin{matrix}
            &   R   \\
            &   0   \\
            & R - 2 \\
            &   R
          \end{matrix}
        \right .
      \right) \\ & \sim \left(
        \begin{matrix}
          1 &   1   &   1   &   1   & \\
          0 & 2 - R &  - R  &  - 2  & \\
          0 &   2   & 4 - R & 2 - R & \\
          0 & R - 2 &   R   &   2   &
        \end{matrix}
        \left|
          \begin{matrix}
            &  1  \\
            & - R \\
            &  0  \\
            &  R
          \end{matrix}
        \right .
      \right) \\ & \sim \left(
        \begin{matrix}
          1 &   1   &  1  &   1   & \\
          0 & 2 - R & - R &  - 2  & \\
          0 &   R   &  4  & 4 - R & \\
          0 &   0   &  0  &   0   &
        \end{matrix}
        \left|
          \begin{matrix}
            &  1  \\
            & - R \\
            &  R  \\
            &  0
          \end{matrix}
        \right .
      \right) \\ & \sim \left(
        \begin{matrix}
          1 &   1   &  1  &   1   & \\
          0 & 1  & 2 - \frac{R}{2} &  1 - \frac{R}{2}  & \\
          0 &   R   &  4  & 4 - R & \\
          0 &   0   &  0  &   0   &
        \end{matrix}
        \left|
          \begin{matrix}
            & 1 \\
            & 0 \\
            & R \\
            & 0
          \end{matrix}
        \right .
      \right) \\ & \sim \left(
        \begin{matrix}
          1 &   1   &  1  &   1   & \\
          0 & 1  & 2 - \frac{R}{2} &  1 - \frac{R}{2}  & \\
          0 &   0   &  4 - 2R + \frac{R^2}{2}  & 4 - 2R + \frac{R^2}{2} & \\
          0 &   0   &  0  &   0   &
        \end{matrix}
        \left|
          \begin{matrix}
            & 1 \\
            & 0 \\
            & R \\
            & 0
          \end{matrix}
        \right .
      \right)
    \end{split}
  \end{equation*}

  Como $4 - 2R + \displaystyle\frac{R^2}{2} \ne 0$ para valores de $R$ enteros entre $2$ y $9$, se tiene que $\rank{(M)} = \rank{(M|N)} = 3$.

  En conclusión, los años \textit{muy buenos} del siglo XXI son: $2002$, $2013$, $2020$, $2024$, $2035$, $2046$, $2057$, $2068$ y $2079$.

\end{document}
