\documentclass[../../main.tex]{subfiles}

\begin{document}

  \begin{shaded}
    Sea $A$ una matriz real de tamaño $4 \times 2$ y $B$ otra matriz real de tamaño $2 \times 4$ tales que:
    $$
    A B =
    \begin{pmatrix}
      1 &  0 & -1 &  0 \\
      0 &  1 &  0 & -1 \\
      -1 &  0 &  1 &  0 \\
      0 & -1 &  0 &  1
    \end{pmatrix}
    $$

    Hallar $B A$.
  \end{shaded}

  \textbf{Solución:}

  Se puede expresar el producto como $
  A B =
  \begin{pmatrix}
    I & -I \\
    -I &  I
  \end{pmatrix}
  $, siendo $I$ la matriz identidad de orden $2$. Y también se puede decir que $A = \begin{pmatrix} A_1 \\ A_2 \end{pmatrix}$ y $B = \begin{pmatrix} B_1 & B_2 \end{pmatrix}$, con $A_1$, $A_2$, $B_1$, $B_2$ matrices $2 \times 2$.

  Entonces,
  $$
  A B =
  \begin{pmatrix} A_1 \\ A_2 \end{pmatrix}
  \begin{pmatrix} B_1 & B_2 \end{pmatrix} =
  \begin{pmatrix}
    I & -I \\
    -I &  I
  \end{pmatrix}
  \iff
  \left \{
  \begin{matrix}
    A_1 B_1 & = &  I \\
    A_1 B_2 & = & -I \\
    A_2 B_1 & = & -I \\
    A_2 B_2 & = &  I \\
  \end{matrix}
  \right .
  $$

  Por otro lado, $B A = \begin{pmatrix} B_1 & B_2 \end{pmatrix} \begin{pmatrix} A_1 \\ A_2 \end{pmatrix} = B_1 A_1 + B_2 A_2$. Como $A_1 B_1 = I$, se puede decir que $B_1 A_1 = I$. Y entonces también $A_2 B_2 = I \iff B_2 A_2 = I$. Visto esto, se concluye que
  $$
  B A = B_1 A_1 + B_2 A_2 = 2 I =
  \begin{pmatrix}
    2 & 0 \\
    0 & 2
  \end{pmatrix}
  $$

\end{document}
