\documentclass[../../main.tex]{subfiles}

\begin{document}

  \begin{shaded}
    Sea $A$ una matriz real $n \times n$ tal que $A^3 = 0$.

    \begin{enumerate}[(a)]
      \item Prueba que existe una única matriz $X$ real $n \times n$ que satisface la ecuación
      $$
      X + A X + X A^2 = A
      $$
      \item Expresa $X$ en términos de $A$.
    \end{enumerate}
  \end{shaded}

  \textbf{Solución:}

  Supóngase que existe una matriz $Y \ne X$ tal que $Y + A Y + Y A^2 = A$. Como $X$ también cumple la ecuación, se tiene que
  $$
  X + A X + X A^2 - (Y + A Y + Y A^2) = 0 \iff (X - Y) + A (X - Y) + (X - Y) A^2 = 0
  $$

  Sea $Z = X - Y$. Se sigue entonces que $Z \ne 0$ ya que $X \ne Y$. Entonces
  \begin{equation*}
    \begin{split}
      Z + A Z + Z A^2 = 0 \Longrightarrow
      A^2 (Z + A Z + Z A^2) A = 0 \\ \iff
      A^2 Z A + A^3 Z A + A^2 Z A^3 = 0 \\ \iff
      A^2 Z A = 0
    \end{split}
  \end{equation*}
  \begin{equation*}
    \begin{split}
      Z + A Z + Z A^2 = 0 \Longrightarrow
      A (Z + A Z + Z A^2) A = 0 \\ \iff
      A Z A + A^2 Z A + A Z A^3 = 0 \\ \iff
      A Z A = 0
    \end{split}
  \end{equation*}
  \begin{equation*}
    \begin{split}
      Z + A Z + Z A^2 = 0 \Longrightarrow
      (Z + A Z + Z A^2) A = 0 \\ \iff
      Z A + A Z A + Z A^3 = 0 \\ \iff
      Z A = 0 \Longrightarrow Z A^2 = 0
    \end{split}
  \end{equation*}
  \begin{equation*}
    \begin{split}
      Z + A Z + Z A^2 = 0 \Longrightarrow
      A^2 (Z + A Z + Z A^2) = 0 \\ \iff
      A^2 Z + A^3 Z + A^2 Z A^2 = 0 \\ \iff
      A^2 Z = 0
    \end{split}
  \end{equation*}
  \begin{equation*}
    \begin{split}
      Z + A Z + Z A^2 = 0 \Longrightarrow
      A (Z + A Z + Z A^2) = 0 \\ \iff
      A Z + A^2 Z + A Z A^2 = 0 \\ \iff
      A Z = 0
    \end{split}
  \end{equation*}

  Entonces, hemos concluido que $A Z = 0$ y que $Z A^2 = 0$. Y por tanto
  $$
  Z + A Z + Z A^2 = 0 \iff Z = 0
  $$

  Contradicción. Entonces ocurre que $X = Y$, y por tanto, la matriz $X$ que cumple la ecuación es única.

  Se puede seguir un procedimiento similar para hallar $X$:
  \begin{equation*}
    \begin{split}
      X + A X + X A^2 = A \Longrightarrow
      A^2 (X + A X + X A^2) A = A^4 = 0 \\ \iff
      A^2 X A + A^3 X A + A^2 X A^3 = 0 \\ \iff
      A^2 X A = 0
    \end{split}
  \end{equation*}
  \begin{equation*}
    \begin{split}
      X + A X + X A^2 = A \Longrightarrow
      A (X + A X + X A^2) A = A^3 = 0 \\ \iff
      A X A + A^2 X A + A X A^3 = 0 \\ \iff
      A X A = 0
    \end{split}
  \end{equation*}
  \begin{equation*}
    \begin{split}
      X + A X + X A^2 = A \Longrightarrow
      (X + A X + X A^2) A = A^2 \\ \iff
      X A + A X A + X A^3 = A^2 \\ \iff
      X A = A^2 \Longrightarrow X A^2 = A^3 = 0
    \end{split}
  \end{equation*}
  \begin{equation*}
    \begin{split}
      X + A X + X A^2 = A \Longrightarrow
      A^2 (X + A X + X A^2) = A^3 = 0 \\ \iff
      A^2 X + A^3 X + A^2 X A^2 = 0 \\ \iff
      A^2 X = 0
    \end{split}
  \end{equation*}
  \begin{equation*}
    \begin{split}
      X + A X + X A^2 = A \Longrightarrow
      A (X + A X + X A^2) = A^2 \\ \iff
      A X + A^2 X + A X A^2 = A^2 \\ \iff
      A X = A^2
    \end{split}
  \end{equation*}

  Y así, $A X = A^2$ y $X A^2 = 0$, por lo que
  $$
  X + A X + X A^2 = A \Longrightarrow X + A^2 = A \Longrightarrow X = A - A^2
  $$

  Otra manera de verlo es definir $X = a I + b A + c A^2$, con $a, b, c \in \R$. De manera que
  \begin{equation*}
    \begin{split}
      X + A X + X A^2 & =
      a I + b A + c A^2 + a A + b A^2 + c A^3 + a A^2 + b A^3 + c A^4 = \\ & =
      a I + b A + c A^2 + a A + b A^2 + a A^2 = \\ & =
      a I + (a + b) A + (a + b + c) A^2 = \\ & = A \iff
      \left \{
        \begin{matrix}
          a &   &   &   &   & = & 0 \\
          a & + & b &   &   & = & 1 \\
          a & + & b & + & c & = & 0 \\
        \end{matrix}
      \right . \iff \left \{
        \begin{matrix}
          a & = &  0 \\
          b & = &  1 \\
          c & = & -1 \\
        \end{matrix}
      \right .
    \end{split}
  \end{equation*}

  Y finalmente queda que $X = A - A^2$.

\end{document}
