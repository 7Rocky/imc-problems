\documentclass[../../main.tex]{subfiles}

\begin{document}

  \begin{shaded}
    Sea $0 < a < b$. Prueba que:
    $$
    \int_a^b (x^2 + 1) \, \e^{-x^2} \dx \geq \e^{-a^2} - \e^{-b^2}
    $$
  \end{shaded}

  \textbf{Solución:}

  Se puede ver fácilmente que:
  $$
  \int_{a^2}^{b^2} \e^{-t} \dt = \e^{-a^2} - \e^{-b^2}
  $$

  Y haciendo un cambio de variable $t = x^2$:
  $$
  \int_a^b (x^2 + 1) \, \e^{-x^2} \dx = \int_{a^2}^{b^2} (t + 1) \, \e^{-t} \, \frac{\text{d}t}{2 \sqrt{t}}
  $$

  Entonces:
  \begin{equation*}
    \begin{split}
      \int_a^b (x^2 + 1) \, \e^{-x^2} \dx \geq \e^{-a^2} - \e^{-b^2} & \iff
      \int_{a^2}^{b^2} (t + 1) \, \e^{-t} \, \frac{\text{d}t}{2 \sqrt{t}} \geq \int_{a^2}^{b^2} \e^{-t} \dt \\ & \iff
      \frac{t + 1}{2 \sqrt{t}} \geq 1 \\ & \iff
      \frac{t^2 + 2t + 1}{4t} \geq 1 \\ & \iff
      (t - 1)^2 \geq 0
    \end{split}
  \end{equation*}

  Como esta desigualdad se cumple para todo $t \in \R$, se demuestra que la desigualdad del enunciado es cierta.

\end{document}
