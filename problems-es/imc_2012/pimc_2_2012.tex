\documentclass[../../main.tex]{subfiles}

\begin{document}

  \begin{shaded}
    Sea $n$ un número natural. Determina el mínimo rango posible de una matrix $n \times n$ que tiene ceros en todas las posiciones la diagonal principal y números positivos en las demás posiciones.
  \end{shaded}

  \textbf{Solución:}

  Para $n = 1$ la única matriz posible es $A_1 = \begin{pmatrix} 0 \end{pmatrix}$, cuyo rango es $\rank{(A_1)} = 0$. Para $n = 2$, se tienen matrices de estilo de
  $$
  A_2 = \begin{pmatrix} 0 & m \\ n & 0 \end{pmatrix}, m, n > 0 \Longrightarrow |A_2| = -m n \ne 0 \Longrightarrow \rank{(A_2)} = 2
  $$

  Para $n \geq 3$ se puede ver que la matriz
  \begin{equation*}
    \begin{split}
      A_n & = \begin{pmatrix}
        0^2 & 1^2 & \dots & (n - 1)^2 \\
        (-1)^2 & 0^2 & \dots & (n - 2)^2 \\
        \vdots & \vdots & \ddots & \vdots \\
        (1 - n)^2 & (2 - n)^2 & \dots & 0^2
      \end{pmatrix} =
      \begin{pmatrix} (i - j)^2 \end{pmatrix}_{i, j = 1}^n =
      \begin{pmatrix} i^2 - 2ij + j^2 \end{pmatrix}_{i, j = 1}^n
      = \\ & =
      \begin{pmatrix}
        1^2 & 1^2 & \dots & 1^2 \\
        2^2 & 2^2 & \dots & 2^2 \\
        \vdots & \vdots & \ddots & \vdots \\
        n^2 & n^2 & \dots & n^2
      \end{pmatrix} -
      2 \cdot \begin{pmatrix}
        1 \cdot 1 & 1 \cdot 2 & \dots & 1 \cdot n \\
        2 \cdot 1 & 2 \cdot 2 & \dots & 2 \cdot n \\
        \vdots & \vdots & \ddots & \vdots \\
        n \cdot 1 & n \cdot 2 & \dots & n \cdot n
      \end{pmatrix} +
      \begin{pmatrix}
        1^2 & 2^2 & \dots & n^2 \\
        1^2 & 2^2 & \dots & n^2 \\
        \vdots & \vdots & \ddots & \vdots \\
        1^2 & 2^2 & \dots & n^2
      \end{pmatrix}
      = \\ & =
      \begin{pmatrix}
        1^2 & 1^2 & \dots & 1^2 \\
        4 & 4 & \dots & 4 \\
        \vdots & \vdots & \ddots & \vdots \\
        n^2 & n^2 & \dots & n^2
      \end{pmatrix} -
      \begin{pmatrix}
        2 & 4 & \dots & 2n \\
        4 & 8 & \dots & 4n \\
        \vdots & \vdots & \ddots & \vdots \\
        2n & 4n & \dots & 2n^2
      \end{pmatrix} +
      \begin{pmatrix}
        1 & 4 & \dots & n^2 \\
        1 & 4 & \dots & n^2 \\
        \vdots & \vdots & \ddots & \vdots \\
        1 & 4 & \dots & n^2
      \end{pmatrix}
    \end{split}
  \end{equation*}

  Y entonces el rango mínimo posible de $A_n$ es
  \begin{equation*}
    \begin{split}
      \rank{(A_n)} & =
      \rank{\left(\begin{pmatrix}
        1^2 & 1^2 & \dots & 1^2 \\
        4 & 4 & \dots & 4 \\
        \vdots & \vdots & \ddots & \vdots \\
        n^2 & n^2 & \dots & n^2
      \end{pmatrix} -
      \begin{pmatrix}
        2 & 4 & \dots & 2n \\
        4 & 8 & \dots & 4n \\
        \vdots & \vdots & \ddots & \vdots \\
        2n & 4n & \dots & 2n^2
      \end{pmatrix} +
      \begin{pmatrix}
        1 & 4 & \dots & n^2 \\
        1 & 4 & \dots & n^2 \\
        \vdots & \vdots & \ddots & \vdots \\
        1 & 4 & \dots & n^2
      \end{pmatrix}\right)} \leq \\ & \leq
      \rank{\begin{pmatrix}
        1^2 & 1^2 & \dots & 1^2 \\
        4 & 4 & \dots & 4 \\
        \vdots & \vdots & \ddots & \vdots \\
        n^2 & n^2 & \dots & n^2
      \end{pmatrix}} +
      \rank{\begin{pmatrix}
        -2 & -4 & \dots & -2n \\
        -4 & -8 & \dots & -4n \\
        \vdots & \vdots & \ddots & \vdots \\
        -2n & -4n & \dots & -2n^2
      \end{pmatrix}} + \\ & +
      \rank{\begin{pmatrix}
        1 & 4 & \dots & n^2 \\
        1 & 4 & \dots & n^2 \\
        \vdots & \vdots & \ddots & \vdots \\
        1 & 4 & \dots & n^2
      \end{pmatrix}} = \\ & = 1 + 1 + 1 = 3
    \end{split}
  \end{equation*}

  Por tanto, el rango mínimo de $A_n$ con $n \geq 3$ es $3$.

\end{document}
