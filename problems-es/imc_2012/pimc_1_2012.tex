\documentclass[../../main.tex]{subfiles}

\begin{document}

  \begin{shaded}
    Para todo número entero positivo $n$, sea $p(n)$ el número de maneras en las que se puede escribir dicho número $n$ como suma de enteros positivos. Por ejemplo, $p(4) = 5$:
    $$
    4 = 3 + 1 = 2 + 2 = 2 + 1 + 1 = 1 + 1 + 1 + 1
    $$

    Se define también $p(0) = 1$. Probar que $p(n) - p(n - 1)$ indica el número de maneras en las que se puede escribir $n$ como suma de enteros estrictamente mayores que $1$.
  \end{shaded}

  \textbf{Solución:}

  Si se supone que $p(n - 1) = x$, entonces $x$ es el número de maneras en las que se puede escribir $n - 1$. Para calcular $p(n)$, puede uno darse cuenta de que $n = (n - 1) + 1$. Como $n - 1$ puede escribirse de $x$ maneras, todas esas $x$ maneras tendrán al menos un término igual a $1$. Entonces, aquellas sumas cuyos sumandos sean mayores que $1$ se denotarán como $y$, de tal manera que $p(n) = x + y$, concluyendo que $y = p(n) - p(n - 1)$.

  Por ejemplo, $p(5) = 7$:
  $$
  5 = (4) + 1 = 3 + 2 = (3 + 1) + 1 = (2 + 2) + 1 = (2 + 1 + 1) + 1 = (1 + 1 + 1 + 1) + 1
  $$

  Se puede observar que las expresiones entre paréntesis son las mismas que aparecen en el ejemplo del enunciado. En este caso, $x = p(4) = 5$, e $y = 2$, de manera que $p(5) = 5 + 2 = 7$.\\

  Siendo un poco más rigurosos, se pueden definir las particiones $\mathcal{P}_n$ para definir el número de particiones de $n$, y $\mathcal{Q}_n$ para definir el número de particiones de $n$ con al menos un sumando igual a $1$. Así, $p(n) = |\mathcal{P}_n|$. Entonces
  $$
  \mathcal{P}_n = \left\{(\lambda_1, \dots, \lambda_k), k \in \N : \lambda_1 \geq \dots \geq \lambda_k, \sum_{i = 1}^k \lambda_i = n \right\}
  $$
  \begin{equation*}
    \begin{split}
      \mathcal{Q}_n & =
      \left\{(\lambda_1, \dots, \lambda_k), k \in \N : \lambda_1 \geq \dots \geq \lambda_k = 1, \sum_{i = 1}^k \lambda_i = n \right\} = \\ & =
      \left\{(\lambda_1, \dots, \lambda_{k - 1}, 1), k \in \N : \lambda_1 \geq \dots \geq \lambda_{k - 1}, \sum_{i = 1}^{k - 1} \lambda_i = n - 1 \right\}
    \end{split}
  \end{equation*}

  Nótese que el número de particiones de $\mathcal{Q}_n$ es igual al número de particiones de $\mathcal{P}_{n - 1}$, ya que
  $$
  \mathcal{P}_{n - 1} = \left\{(\lambda_1, \dots, \lambda_{k - 1}), k \in \N : \lambda_1 \geq \dots \geq \lambda_{k - 1}, \sum_{i = 1}^{k - 1} \lambda_i = n - 1 \right\}
  $$

  Por lo que $|\mathcal{Q}_{n}| = |\mathcal{P}_{n - 1}|$. Entonces, el conjunto de particiones de $n$ con todos los sumandos estrictamente mayores que $1$ será $\mathcal{P}_{n} \setminus \mathcal{Q}_{n}$, de manera que el número de elementos del conjunto será
  $$
  |\mathcal{P}_{n} \setminus \mathcal{Q}_{n}| = |\mathcal{P}_{n}| - |\mathcal{Q}_{n}| = |\mathcal{P}_{n}| - |\mathcal{P}_{n - 1}| = p(n) - p(n - 1)
  $$

  Esto se debe demostrar más rigurosamente definiendo un mapa biyectivo entre los conjuntos $\mathcal{P}_{n-1}$ y $\mathcal{Q}_{n}$, de manera que
  $$
  \varphi : \mathcal{P}_{n-1} \to \mathcal{Q}_{n} \quad / \quad \varphi(\lambda_1, \dots, \lambda_k) = (\lambda_1, \dots, \lambda_k, 1)
  $$

  Para que este mapa sea biyectivo, debe ser inyectivo y sobreyectivo.

  El mapa $\varphi$ es inyectivo ya que $\varphi(\bar{v}) = \varphi(\bar{w}) \iff (\bar{v}, 1) = (\bar{w}, 1) \iff \bar{v} = \bar{w}$. Y es sobreyectivo ya que $\forall \, \bar{V} = (\bar{v}, 1) \in \mathcal{Q}_n \; \exists \,\bar{v} \in \mathcal{P}_{n - 1} : \varphi(\bar{v}) = (\bar{v}, 1) = \bar{V}$.

\end{document}
